\documentclass[14pt, a4paper]{article} % 文档类
\usepackage[UTF8]{ctex} % 支持中文
\usepackage{amsmath, amssymb, amsthm} % 数学公式和定理环境
\usepackage{graphicx} % 插入图片
\usepackage{float} % 图片位置控制
\usepackage{subcaption} % 子图
\usepackage{hyperref} % 超链接
\usepackage{listings} % 代码环境
\usepackage{xcolor} % 颜色
\usepackage{booktabs} % 三线表
\usepackage{multirow} % 合并表格行
\usepackage{multicol} % 多栏排版
\usepackage{geometry} % 设置页边距
\usepackage{fancyhdr} % 页眉页脚
\usepackage{titlesec} % 标题格式
\usepackage{titletoc} % 目录格式
\usepackage{cite} % 参考文献
\usepackage{enumerate} % 枚举环境
\usepackage{enumitem} % 定制列表
\usepackage{ulem} % 下划线等
\usepackage{indentfirst} % 首行缩进
\usepackage{physics} % 物理宏包
\usepackage{mathrsfs}
% 页面设置
\geometry{left=1.5cm, right=1.5cm, top=1.5cm, bottom=1.5cm}
\setlength{\parindent}{2em} % 首行缩进2字符
\graphicspath{{./}}
% 代码样式设置
\lstset{
	language=Fortran,
	basicstyle=\ttfamily\small,
	keywordstyle=\color{blue},
	commentstyle=\color{green!50!black},
	stringstyle=\color{red},
	numbers=left,
	numberstyle=\tiny\color{gray},
	frame=single,
	breaklines=true,
	captionpos=b
}
% 页眉页脚设置
\pagestyle{fancy}
\fancyhf{}
\fancyhead[C]{\leftmark}
\fancyfoot[C]{\thepage}
% 标题格式设置
\titleformat{\section}{\Large\bfseries}{\thesection}{1em}{}
\titleformat{\subsection}{\large\bfseries}{\thesubsection}{1em}{}
% 文档信息
\title{\textbf{ZZH}}
\author{张桢 \\ \texttt{swip\_zhang\_zhen@foxmail.com}}
\date{\today}
\begin{document}
	\maketitle % 生成标题
	\begin{abstract}
		\dots	
		\textbf{关键词:} ZZH
	\end{abstract}
	\tableofcontents % 生成目录
	\newpage
    \section{note}
    \subsection{note}
    \subsubsection{note}
    \begin{gather}
        double\ m\& l \\
        t\ \theta(t)\ \varphi(t) \\
        m_1=m_2=m \\
        l_1=l_2=l \\
        for\ m_1 \\
        T_1=\dfrac{1}{2}ml^2\dot{\theta}^2 \\
        V_1=-mgl\cos(\theta) \\
        L_1=T_1-V_1 \\
        for\ m_2 \\
        V_2=-mgl(\cos(\theta)+\cos(\varphi)) \\
        T_2=\dfrac{1}{2}ml^2[\dot{\theta}^2+\dot{\varphi}^2+2\dot{\theta}\dot{\varphi}\cos(\theta-\varphi)] \\
        L_2=T_2-V_2 \\
        L=\dfrac{1}{2}ml^2[2\dot{\theta}^2+\dot{\varphi}^2+2\dot{\theta}\dot{\varphi}\cos(\theta-\varphi)]+mgl[2\cos(\theta)+\cos(\varphi)] \\
        \dfrac{d}{dt}(\dfrac{\partial L}{\partial\dot{\theta}})=\dfrac{\partial L}{\partial\theta}\  \rightarrow \\
        2ml^2\ddot{\theta}+ml^2\ddot{\varphi}\cos(\theta-\varphi)-ml^2\dot{\varphi}\sin(\theta-\varphi)(\dot{\theta}-\dot{\varphi})=-2mgl\sin(\theta)-ml^2\dot{\theta}\dot{\varphi}\sin(\theta-\varphi) \\
        2\ddot{\theta}+\ddot{\varphi}\cos(\theta-\varphi)+\dot{\varphi}^2\sin(\theta-\varphi)+2\dfrac{g}{l}\sin(\theta)=0 \\
        \dfrac{d}{dt}(\dfrac{\partial L}{\partial \dot{\varphi}})=\dfrac{\partial L}{\partial\varphi}\ \rightarrow \\
        ml^2\ddot{\varphi}+ml^2\ddot{\theta}\cos(\theta-\varphi)-ml^2\dot{\theta}\sin(\theta-\varphi)(\dot{\theta}-\dot{\varphi})=-mgl\sin(\varphi)+ml^2\dot{\theta}\dot{\varphi}\sin(\theta-\varphi) \\
        \ddot{\varphi}+\ddot{\theta}\cos(\theta-\varphi)-\dot{\theta}^2\sin(\theta-\varphi)+\dfrac{g}{l}\sin(\varphi)=0 \\
        \rightarrow \\
        \ddot{\varphi}+\ddot{\theta}\cos(\theta-\varphi)-\dot{\theta}^2\sin(\theta-\varphi)+\dfrac{g}{l}\sin(\varphi)=0 \\
        2\ddot{\theta}+\ddot{\varphi}\cos(\theta-\varphi)+\dot{\varphi}^2\sin(\theta-\varphi)+2\dfrac{g}{l}\sin(\theta)=0 \\
        \vec{y}=\begin{pmatrix}
            \theta \\
            \varphi \\
            \dot{\theta} \\
            \dot{\varphi} \\
        \end{pmatrix}
        =\begin{pmatrix}
            y_1 \\
            y_2 \\
            y_3 \\
            y_4 \\
        \end{pmatrix} \\
        \dot{\vec{y}}=\begin{pmatrix}
            y_3 \\
            y_4 \\
            \dfrac{1}{\cos^2(y_1-y_2)-2}[\sin(y_1-y_2)(\cos(y_1-y_2)y_3^2+y_4^2)+\dfrac{g}{l}(2\sin y_1-\cos(y_1-y_2)\sin y_2)] \\
            \dfrac{1}{\cos^2(y_1-y_2)-2}[-\sin(y_1-y_2)(2y_3^2+\cos(y_1-y_2)y_4^2)+\dfrac{g}{l}(2\sin y_2-2\cos(y_1-y_2)\sin y_1)]
        \end{pmatrix}
    \end{gather}
    \begin{flushleft}
        $$\vec{y}=\begin{pmatrix}
            \theta \\
            \varphi \\
            \dot{\theta} \\
            \dot{\varphi} \\
        \end{pmatrix}
        =\begin{pmatrix}
            y_1 \\
            y_2 \\
            y_3 \\
            y_4 \\
        \end{pmatrix};
        \dot{\vec{y}}=\begin{pmatrix}
            y_3 \\
            y_4 \\
            \dfrac{1}{\cos^2(y_1-y_2)-2}[\sin(y_1-y_2)(\cos(y_1-y_2)y_3^2+y_4^2)+\dfrac{g}{l}(2\sin y_1-\cos(y_1-y_2)\sin y_2)] \\
            \dfrac{1}{\cos^2(y_1-y_2)-2}[-\sin(y_1-y_2)(2y_3^2+\cos(y_1-y_2)y_4^2)+\dfrac{g}{l}(2\sin y_2-2\cos(y_1-y_2)\sin y_1)]
        \end{pmatrix}$$
    \end{flushleft}
    \begin{flushright}
        end
    \end{flushright}
    \begin{lstlisting}
    module constants_module
    implicit none
    public
    
    ! 物理常量
    real(8), parameter :: g = 9.81d0          ! 重力加速度 (m/s^2)
    real(8), parameter :: l = 1.0d0           ! 摆长 (m)
    real(8), parameter :: m = 1.0d0           ! 质量 (kg)
    
    ! 模拟参数
    real(8), parameter :: t_end = 20.0d0      ! 结束时间 (s)
    real(8), parameter :: dt = 0.001d0        ! 时间步长 (s)
    integer, parameter :: n = 4               ! 系统维度
    integer, parameter :: n_steps = nint(t_end / dt)  ! 总步数
    
    ! 初始条件 (θ, φ, θ_dot, φ_dot)
    real(8), parameter :: y0(4) = [0.1d0, 0.2d0, 0.0d0, 0.0d0]
    
    end module constants_module
    \end{lstlisting}
    \begin{lstlisting}
    module rk4_solver_module
    use constants_module, only: n, dt, g, l
    implicit none
    private
    public :: rk4_step
    
    contains
    
    ! 双摆微分方程 - 严格按你的公式
    subroutine derivs(t, y, dydt)
        real(8), intent(in) :: t
        real(8), intent(in) :: y(n)        ! y = [θ, φ, θ_dot, φ_dot]
        real(8), intent(out) :: dydt(n)
        
        real(8) :: theta, phi, theta_dot, phi_dot
        real(8) :: diff, cos_diff, sin_diff, denom
        
        ! 提取变量
        theta = y(1)     ! θ
        phi = y(2)       ! φ
        theta_dot = y(3) ! θ_dot
        phi_dot = y(4)   ! φ_dot
        
        ! 角度差
        diff = theta - phi
        cos_diff = cos(diff)
        sin_diff = sin(diff)
        
        ! 分母 cos^2(θ-φ) - 2
        denom = cos_diff**2 - 2.0d0
        
        ! 状态方程
        dydt(1) = theta_dot  ! dθ/dt = θ_dot
        dydt(2) = phi_dot    ! dφ/dt = φ_dot
        
        ! dθ_dot/dt - 严格按照你的公式
        dydt(3) = (1.0d0 / denom) * &
                 (sin_diff * (cos_diff * theta_dot**2 + phi_dot**2) + &
                 (g/l) * (2.0d0 * sin(theta) - cos_diff * sin(phi)))
        
        ! dφ_dot/dt - 严格按照你的公式
        dydt(4) = (1.0d0 / denom) * &
                 (-sin_diff * (2.0d0 * theta_dot**2 + cos_diff * phi_dot**2) + &
                 (g/l) * (2.0d0 * sin(phi) - 2.0d0 * cos_diff * sin(theta)))
        
    end subroutine derivs
    
    ! 经典四阶龙格库塔单步推进
    subroutine rk4_step(t, y, y_next)
        real(8), intent(in) :: t
        real(8), intent(in) :: y(n)
        real(8), intent(out) :: y_next(n)
        
        real(8) :: k1(n), k2(n), k3(n), k4(n)
        real(8) :: y_temp(n)
        
        ! 第一步
        call derivs(t, y, k1)
        
        ! 第二步
        y_temp = y + 0.5d0 * dt * k1
        call derivs(t + 0.5d0 * dt, y_temp, k2)
        
        ! 第三步
        y_temp = y + 0.5d0 * dt * k2
        call derivs(t + 0.5d0 * dt, y_temp, k3)
        
        ! 第四步
        y_temp = y + dt * k3
        call derivs(t + dt, y_temp, k4)
        
        ! 组合
        y_next = y + (dt / 6.0d0) * (k1 + 2.0d0 * k2 + 2.0d0 * k3 + k4)
        
    end subroutine rk4_step
    
    end module rk4_solver_module
    \end{lstlisting}
    \begin{lstlisting}
    module energy_module
    use constants_module, only: n, g, l, m
    implicit none
    private
    public :: compute_energy
    
    contains
    
    ! 计算系统能量
    subroutine compute_energy(y, kinetic, potential, total)
        real(8), intent(in) :: y(n)
        real(8), intent(out) :: kinetic, potential, total
        
        real(8) :: theta, phi, theta_dot, phi_dot
        real(8) :: diff, cos_diff
        
        ! 提取变量
        theta = y(1)
        phi = y(2)
        theta_dot = y(3)
        phi_dot = y(4)
        
        diff = theta - phi
        cos_diff = cos(diff)
        
        ! 动能
        kinetic = 0.5d0 * m * l**2 * &
                  (2.0d0 * theta_dot**2 + phi_dot**2 + &
                   2.0d0 * theta_dot * phi_dot * cos_diff)
        
        ! 势能
        potential = -m * g * l * (2.0d0 * cos(theta) + cos(phi))
        
        ! 总能量
        total = kinetic + potential
        
    end subroutine compute_energy
    
    end module energy_module
    \end{lstlisting}
    \begin{lstlisting}
    module output_module
    use constants_module, only: n, dt, t_end, n_steps
    use energy_module, only: compute_energy
    implicit none
    private
    public :: write_results
    
    contains
    
    ! 写入结果文件
    subroutine write_results(time_array, state_array, filename)
        real(8), intent(in) :: time_array(n_steps)
        real(8), intent(in) :: state_array(n_steps, n)
        character(len=*), intent(in) :: filename
        
        integer :: i, unit
        real(8) :: kinetic, potential, total
        
        open(newunit=unit, file=filename, status='replace')
        
        ! 写入表头
        write(unit, '(A)') 'Time(s)    Theta(rad)    Phi(rad)    '// &
                           'Theta_dot(rad/s)    Phi_dot(rad/s)    '// &
                           'Kinetic(J)    Potential(J)    Total(J)'
        
        ! 写入数据
        do i = 1, n_steps
            ! 计算能量
            call compute_energy(state_array(i,:), kinetic, potential, total)
            
            ! 写入一行数据
            write(unit, '(8ES16.8)') &
                time_array(i), &
                state_array(i,1), &   ! theta
                state_array(i,2), &   ! phi
                state_array(i,3), &   ! theta_dot
                state_array(i,4), &   ! phi_dot
                kinetic, &
                potential, &
                total
        end do
        
        close(unit)
        
        print *, '结果已保存到文件: ', trim(filename)
        
    end subroutine write_results
    
    end module output_module
    \end{lstlisting}
    \begin{lstlisting}
    program double_pendulum_main
    use constants_module, only: n, dt, t_end, n_steps, y0
    use rk4_solver_module, only: rk4_step
    use output_module, only: write_results
    implicit none
    
    real(8) :: time_array(n_steps), state_array(n_steps, n)
    real(8) :: current_time, current_state(n)
    integer :: i
    
    ! 初始化
    current_time = 0.0d0
    current_state = y0
    time_array(1) = current_time
    state_array(1,:) = current_state
    
    ! 时间推进
    do i = 1, n_steps-1
        ! 单步龙格库塔
        call rk4_step(current_time, current_state, current_state)
        
        ! 更新时间
        current_time = current_time + dt
        
        ! 存储结果
        time_array(i+1) = current_time
        state_array(i+1,:) = current_state
        
        ! 进度显示
        if (mod(i, 1000) == 0) then
            print '(A,F6.1,A,F6.1)', '进度: ', current_time, ' / ', t_end
        end if
    end do
    
    print *, '模拟完成,总步数: ', n_steps
    
    ! 写入结果
    call write_results(time_array, state_array, 'double_pendulum_results.txt')
    
    end program double_pendulum_main
    \end{lstlisting}
    \begin{lstlisting}
# 编译器设置
FC = gfortran
FFLAGS = -O2 -Wall -Wextra
LDFLAGS = 

# 目标文件
OBJS = constants_module.o rk4_solver_module.o energy_module.o output_module.o main.o

# 可执行文件
TARGET = double_pendulum

# 默认目标
all: $(TARGET)

# 链接可执行文件
$(TARGET): $(OBJS)
	$(FC) $(FFLAGS) -o $(TARGET) $(OBJS) $(LDFLAGS)

# 编译主程序
main.o: main.f90 constants_module.o rk4_solver_module.o output_module.o
	$(FC) $(FFLAGS) -c main.f90

# 编译输出模块
output_module.o: output_module.f90 constants_module.o energy_module.o
	$(FC) $(FFLAGS) -c output_module.f90

# 编译能量模块
energy_module.o: energy_module.f90 constants_module.o
	$(FC) $(FFLAGS) -c energy_module.f90

# 编译求解器模块
rk4_solver_module.o: rk4_solver_module.f90 constants_module.o
	$(FC) $(FFLAGS) -c rk4_solver_module.f90

# 编译常量模块
constants_module.o: constants_module.f90
	$(FC) $(FFLAGS) -c constants_module.f90

# 清理
clean:
	rm -f *.o *.mod $(TARGET)

# 运行
run: $(TARGET)
	./$(TARGET)

# 查看结果
check:
	@echo "前10行结果:"
	@head -n 11 double_pendulum_results.txt

.PHONY: all clean run check
    \end{lstlisting}
    \begin{lstlisting}
% 读取双摆模拟结果文件
clear; clc;
cd 'C://Users/wudou/zhz/zhz/clt/zzh/';
% 设置文件名
filename = './data/double_pendulum_results.txt';
% 检查文件是否存在
if ~exist(filename, 'file')
    error('文件不存在: %s\n请确保已经运行了Fortran程序并生成了结果文件。', filename);
end
% 读取数据
fprintf('正在读取文件: %s\n', filename);
% 使用textscan读取文件(跳过第一行表头)
fid = fopen(filename, 'r');
if fid == -1
    error('无法打开文件: %s', filename);
end
% 读取表头
header = fgetl(fid);
% 使用textscan读取数据(科学计数法格式)
data = textscan(fid, '%f %f %f %f %f %f %f %f');
% 关闭文件
fclose(fid);
% 提取各列数据到工作区
time = data{1};              % 时间 (s)
theta = data{2};             % θ角 (rad)
phi = data{3};               % φ角 (rad)
theta_dot = data{4};         % θ角速度 (rad/s)
phi_dot = data{5};          % φ角速度 (rad/s)
kinetic = data{6};           % 动能 (J)
potential = data{7};         % 势能 (J)
total = data{8};             % 总能量 (J)
% 显示基本信息
fprintf('数据读取成功!\n');
fprintf('总数据点数: %d\n', length(time));
fprintf('时间范围: %.2f ~ %.2f s\n', time(1), time(end));
fprintf('θ角范围: %.4f ~ %.4f rad\n', min(theta), max(theta));
fprintf('φ角范围: %.4f ~ %.4f rad\n', min(phi), max(phi));
fprintf('总能量范围: %.6f ~ %.6f J\n', min(total), max(total));
% 计算能量相对误差(检查能量守恒)
energy_error = (max(total) - min(total)) / mean(total) * 100;
fprintf('能量相对误差: %.4e %%\n', energy_error);
% 可选:将数据保存为.mat文件以便后续使用
%save('double_pendulum_data.mat', 'time', 'theta', 'phi', 'theta_dot', 'phi_dot', ...'kinetic', 'potential', 'total', 'header');
%fprintf('数据已保存到工作区变量中,并已保存为 double_pendulum_data.mat 文件。\n');
fprintf('可用变量:\n');
fprintf('  time: 时间序列\n');
fprintf('  theta, phi: 角度\n');
fprintf('  theta_dot, phi_dot: 角速度\n');
fprintf('  kinetic, potential, total: 能量\n');
fprintf('  header: 表头信息\n');
dir='./result/';
figure;
plot(time,theta,'LineWidth',1.5,'Color','b');
title('$$\theta\sim t$$','Interpreter','latex','FontSize',16);
xlabel('$$t$$','Interpreter','latex','FontSize',16);
ylabel('$$\theta$$','Interpreter','latex','FontSize',16);
legend('$$\theta$$','Interpreter','latex','Location','north');
fname=sprintf('theta-t');
print('-dpng',[dir '/' fname]);
close;
figure;
plot(time,phi,'LineWidth',1.5,'Color','b');
title('$$\varphi\sim t$$','Interpreter','latex','FontSize',16);
xlabel('$$t$$','Interpreter','latex','FontSize',16);
ylabel('$$\varphi$$','Interpreter','latex','FontSize',16);
legend('$$varphi$$','Interpreter','latex','Location','north');
fname=sprintf('phi-t');
print('-dpng',[dir '/' fname]);
close;
figure;
plot(time,phi_dot,'LineWidth',1.5,'Color','b');
title('$$\dot{\varphi}\sim t$$','Interpreter','latex','FontSize',16);
xlabel('$$t$$','Interpreter','latex','FontSize',16);
ylabel('$$\dot{\varphi}$$','Interpreter','latex','FontSize',16);
legend('$$\dot{\varphi}$$','Interpreter','latex','Location','north');
fname=sprintf('phi_dot-t');
print('-dpng',[dir '/' fname]);
close;
figure;
plot(time,phi_dot,'LineWidth',1.5,'Color','b');
title('$$\dot{\theta}\sim t$$','Interpreter','latex','FontSize',16);
xlabel('$$t$$','Interpreter','latex','FontSize',16);
ylabel('$$\dot{\theta}$$','Interpreter','latex','FontSize',16);
legend('$$\dot{\theta}$$','Interpreter','latex','Location','north');
fname=sprintf('theta_dot-t');
print('-dpng',[dir '/' fname]);
close;
figure;
plot(time,kinetic,'LineWidth',1.5,'Color','b');
title('$$T\sim t$$','Interpreter','latex','FontSize',16);
xlabel('$$t$$','Interpreter','latex','FontSize',16);
ylabel('$$T$$','Interpreter','latex','FontSize',16);
legend('$$T$$','Interpreter','latex','Location','north');
fname=sprintf('T-t');
print('-dpng',[dir '/' fname]);
close;
figure;
plot(time,potential,'LineWidth',1.5,'Color','b');
title('$$V\sim t$$','Interpreter','latex','FontSize',16);
xlabel('$$t$$','Interpreter','latex','FontSize',16);
ylabel('$$V$$','Interpreter','latex','FontSize',16);
legend('$$V$$','Interpreter','latex','Location','north');
fname=sprintf('V-t');
print('-dpng',[dir '/' fname]);
close;
figure;
plot(time,total,'LineWidth',5,'Color','b');
title('E');

close;

% 设置参数
l = 1.0;  % 摆长
skip_frames = 20;  % 跳过帧数(加速动画)
% 提取部分数据用于动画
indices = 1:skip_frames:length(time);
time_anim = time(indices);
theta_anim = theta(indices);
phi_anim = phi(indices);

% 计算笛卡尔坐标
x1 = l * sin(theta_anim);
y1 = -l * cos(theta_anim);
x2 = x1 + l * sin(phi_anim);
y2 = y1 - l * cos(phi_anim);

% 创建图形窗口
figure('Position', [100, 100, 800, 800]);
axis equal;
xlim([-2.1*l, 2.1*l]);
ylim([-2.1*l, 2.1*l]);
grid on;
hold on;
title('双摆运动动画');
xlabel('x');
ylabel('y');
% 绘制固定点
plot(0, 0, 'ko', 'MarkerSize', 12, 'MarkerFaceColor', 'k');
% 初始化图形对象
pendulum_line1 = plot([0, x1(1)], [0, y1(1)], 'b-', 'LineWidth', 1);
pendulum_line2 = plot([x1(1), x2(1)], [y1(1), y2(1)], 'r-', 'LineWidth', 1);
mass1 = plot(x1(1), y1(1), 'bo', 'MarkerSize', 12, 'MarkerFaceColor', 'b');
mass2 = plot(x2(1), y2(1), 'ro', 'MarkerSize', 12, 'MarkerFaceColor', 'r');
trajectory = plot(x2(1), y2(1), 'g-', 'LineWidth', 1.5, 'Color', [0, 0.7, 0, 0.5]);
% 时间显示文本
time_text = text(-2*l, -0.5*l, sprintf('时间: %.2f s', time_anim(1)), ...
    'FontSize', 14, 'FontWeight', 'bold');
% 轨迹数据存储
trajectory_x = [];
trajectory_y = [];
% 生成动画
fprintf('生成动画...\n');
for i = 1:length(time_anim)
    % 更新摆的位置
    set(pendulum_line1, 'XData', [0, x1(i)], 'YData', [0, y1(i)]);
    set(pendulum_line2, 'XData', [x1(i), x2(i)], 'YData', [y1(i), y2(i)]);
    set(mass1, 'XData', x1(i), 'YData', y1(i));
    set(mass2, 'XData', x2(i), 'YData', y2(i));
    
    % 更新轨迹
    trajectory_x = [trajectory_x, x2(i)];
    trajectory_y = [trajectory_y, y2(i)];
    set(trajectory, 'XData', trajectory_x, 'YData', trajectory_y);
    
    % 更新时间显示
    set(time_text, 'String', sprintf('时间: %.2f s', time_anim(i)));
    
    % 刷新图形
    drawnow;
    
    % 暂停控制动画速度
    pause(0.01);
end
fprintf('动画完成\n');
    \end{lstlisting}
    \begin{figure}[H]  % h: 这里, t: 顶部, b: 底部, p: 独立页
		\centering
		\includegraphics[scale=0.6]{theta-t.png}  % 第一张图
		\caption{theta}
    \end{figure}
    \begin{figure}[H]  % h: 这里, t: 顶部, b: 底部, p: 独立页
		\centering
		\includegraphics[scale=0.6]{phi-t.png}  % 第一张图
		\caption{phi}
    \end{figure}
    \begin{figure}[H]  % h: 这里, t: 顶部, b: 底部, p: 独立页
		\centering
		\includegraphics[scale=0.6]{theta_dot-t.png}  % 第一张图
		\caption{thetadot}
    \end{figure}
    \begin{figure}[H]  % h: 这里, t: 顶部, b: 底部, p: 独立页
		\centering
		\includegraphics[scale=0.6]{phi-t.png}  % 第一张图
		\caption{phi}
    \end{figure}
    \begin{figure}[H]  % h: 这里, t: 顶部, b: 底部, p: 独立页
		\centering
		\includegraphics[scale=0.6]{phi_dot-t.png}  % 第一张图
		\caption{phidot}
    \end{figure}
    \begin{figure}[H]  % h: 这里, t: 顶部, b: 底部, p: 独立页
		\centering
		\includegraphics[scale=0.6]{T-t.png}  % 第一张图
		\caption{T}
    \end{figure}
    \begin{figure}[H]  % h: 这里, t: 顶部, b: 底部, p: 独立页
		\centering
		\includegraphics[scale=0.6]{V-t.png}  % 第一张图
		\caption{V}
    \end{figure}
    \begin{figure}[H]  % h: 这里, t: 顶部, b: 底部, p: 独立页
		\centering
		\includegraphics[scale=0.6]{animation.png}  % 第一张图
		\caption{animation}
    \end{figure}
\end{document}